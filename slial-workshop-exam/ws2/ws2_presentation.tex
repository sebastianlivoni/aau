\documentclass{beamer}

\usepackage[utf8]{inputenc}

\title{Sandsynlighedsteori og Lineær algebra}
\subtitle{Workshop 2 - PageRank}
\author{Sebastian Livoni Larsen}

\begin{document}
	\frame {
		\titlepage
	}
	\frame {
		\frametitle{Delopgave 1}
		\framesubtitle{\textbf{Opgave 2} - Sandsynlighederne $p(X=y)$ for de ikke-tomme urbilleder af $y$.}
    \fontsize{7pt}{7.2}\selectfont  
    \[p(X=0)=p(w_iw_iw_i)=\frac{1}{2} \cdot \frac{1}{2} \cdot \frac{1}{2}=\frac{1}{8}=0.125\]
		\[p(X=1)=p(w_1w_iw_i)+p(w_iw_1w_i)+p(w_iw_iw_1)=\frac{1}{2}\cdot 1 \cdot \frac{1}{2}+\frac{1}{2} \cdot \frac{1}{2} \cdot 1+\frac{1}{2}\cdot \frac{1}{2} \cdot \frac{1}{2}=\frac{5}{8}=0.625 \]  
		\[p(X=2)=p(w_1w_1w_i)+p(w_iw_1w_1)+p(w_1w_iw_1)=\frac{1}{2} \cdot 0 \cdot \frac{1}{2}+\frac{1}{2} \cdot \frac{1}{2} \cdot 0+\frac{1}{2} \cdot 1 \cdot \frac{1}{2}=\frac{2}{8}=0.25\]
		\[p(X=3)=p(w_1w_1w_1)=\frac{1}{2} \cdot 0 \cdot 0=0 \]
	}
  \frame{
    \frametitle{Delopgave 1}
    \framesubtitle{\textbf{Opgave 3} - Middelværdien og variansen for $X$.}
    \fontsize{7pt}{7.2}\selectfont

    \begin{align*}
      \text{Middelværdi: }\begin{aligned}[t] E(X)=\sum_{s\in S} p(s)X(s)&=0.125 \cdot 0+0.625 \cdot 1+0.25 \cdot 2=\mathbf{1.125}
      \end{aligned} 
    \end{align*}
    
    \begin{align*}
      \begin{aligned}[t] V(X) &= \sum_{s \in S} (X(s)-E(X))^2p(s) \\
        &= (0-1.125)^2 \cdot 0.125 + (1-1.125)^2 \cdot 0.625+(2-1.125)^2 \cdot 0.25 = \mathbf{0.359375}
      \end{aligned} 
    \end{align*}
  }
  \frame{
    \frametitle{Delopgave 1}
    \framesubtitle{\textbf{Opgave 4} - Bernoullifordeling med sandsynlighed $\frac{3}{8}$ for success.}
    \fontsize{7pt}{7.2}\selectfont

    \begin{align*}
      \begin{gathered}
        \text{Bernoulli trial}=b(k,n,p) \text{ where $k$=successes, $n$=independent trials, $p$=probability of success} \\
        b(k,n,p)=\text{C}(n,k)p^kq^{n-k} \\
        b\left(0;3;\frac{3}{8}\right)=\frac{3!}{0! \cdot (3-0)!} \cdot \frac{3}{8}^0 \cdot \left(1-\frac{3}{8}\right)^{3-0}=0.244140625 \\
        b\left(1;3;\frac{3}{8}\right)=\frac{3!}{1! \cdot (3-1)!} \cdot \frac{3}{8}^1 \cdot \left(1-\frac{3}{8}\right)^{3-1}=0.439453125 \\
        b\left(2;3;\frac{3}{8}\right)=\frac{3!}{2! \cdot (3-2)!} \cdot \frac{3}{8}^2 \cdot \left(1-\frac{3}{8}\right)^{3-2}=0.263671875 \\
        b\left(3;3;\frac{3}{8}\right)=\frac{3!}{3! \cdot (3-3)!} \cdot \frac{3}{8}^3 \cdot \left(1-\frac{3}{8}\right)^{3-3}=0.052734375 \\
      \end{gathered}
    \end{align*}
    \fontsize{5pt}{7.2}\selectfont
    \begin{align*}
      \text{Middelværdi: }\begin{aligned}[t] E(X)=\sum_{s\in S} p(s)X(s)&=0.244140625 \cdot 0+0.439453125 \cdot 1+0.263671875 \cdot 2 + 0.052734375 \cdot 3=\mathbf{1.125}
      \end{aligned} 
    \end{align*}
    \begin{align*}
      \text{Varians: }\begin{aligned}[t] V(X) &= \sum_{s \in S} (X(s)-E(X))^2p(s) \\
      &= (0-1.125)^2 \cdot 0.244 + (1-1.125)^2 \cdot 0.439+(2-1.125)^2 \cdot 0.263+ (3-1.125)^2 \cdot 0.052 = \mathbf{0.703125}
      \end{aligned} 
    \end{align*}
  }
  \frame {
		\frametitle{Delopgave 2}
		\framesubtitle{\textbf{Opgave 1} - Sandsynligheden for at være i tilstand $w_1$ til tiden $t=5$.}
    \fontsize{9pt}{7.2}\selectfont
    
    \[ S=\begin{bmatrix}\frac{1}{2} & \frac{1}{8} & \frac{1} {8} & \frac{1} {8} & \frac{1} {8}\end{bmatrix}^T\]
		\[P^{(5)}=P^5P^{(0)}=\begin{bmatrix}
		0 & \frac{1}{2} & \frac{1}{2} & \frac{1}{2} & \frac{1}{2}\\
		\frac{1}{4} & \frac{1}{8} & \frac{1}{8} & \frac{1}{8} & \frac{1}{8}\\
		\frac{1}{4} & \frac{1}{8} & \frac{1}{8} & \frac{1}{8} & \frac{1}{8}\\
		\frac{1}{4} & \frac{1}{8} & \frac{1}{8} & \frac{1}{8} & \frac{1}{8}\\
		\frac{1}{4} & \frac{1}{8} & \frac{1}{8} & \frac{1}{8} & \frac{1}{8}
	\end{bmatrix}^5\begin{bmatrix}\frac{1}{2} & \frac{1}{8} & \frac{1} {8} & \frac{1} {8} & \frac{1} {8}\end{bmatrix}^T=\begin{bmatrix}
		\mathbf{0.328125} \\ 0.16796875 \\ 0.16796875 \\ 0.16796875 \\ 0.16796875
	\end{bmatrix}\]
	}
  \frame {
		\frametitle{Delopgave 2}
		\framesubtitle{\textbf{Opgave 2} - Bestem Markov-kædens stationære fordeling hvis den har en.}
    \fontsize{6pt}{7.2}\selectfont

    \begin{itemize}
      \item Ja, den har en stationær fordeling pga. $N \times N$ og ingen negative værdier.
    \end{itemize}

    \[
    (P-I)x=0 \Longleftrightarrow 
    \begin{bmatrix}
      -1 & \frac{1}{2} & \frac{1}{2} & \frac{1}{2} & \frac{1}{2} & 0 \\
      \frac{1}{4} & -\frac{7}{8} & \frac{1}{8} & \frac{1}{8} & \frac{1}{8} & 0 \\
      \frac{1}{4} & \frac{1}{8} & -\frac{7}{8} & \frac{1}{8} & \frac{1}{8} & 0 \\
      \frac{1}{4} & \frac{1}{8} & \frac{1}{8} & -\frac{7}{8} & \frac{1}{8} & 0 \\
      \frac{1}{4} & \frac{1}{8} & \frac{1}{8} & \frac{1}{8} & -\frac{7}{8} & 0 \\
    \end{bmatrix}
    =
    \begin{bmatrix}
      1 & 0 & 0 & 0 & -2 & 0 \\
      0 & 1 & 0 & 0 & -1 & 0 \\
      0 & 0 & 1 & 0 & -1 & 0 \\
      0 & 0 & 0 & 1 & -1 & 0 \\
      0 & 0 & 0 & 0 & 0 & 0 \\
    \end{bmatrix} = \begin{matrix} x_1=2x_5 \\x_2=1x_5 \\x_3 =1x_5 \\ x_4=1x_5 \\x_5 \text{ is free} \end{matrix} \]
    \[ x_5= \begin{bmatrix}2 \\ 1 \\ 1 \\ 1 \\ 1 \end{bmatrix} \]
    \[q=\begin{bmatrix}2/6 \\ 1/6 \\ 1/6 \\ 1/6 \\ 1/6
    \end{bmatrix}\]

    \begin{itemize}
      \item Større sandsynlighed for at befinde sig på side 1.
    \end{itemize}
	}
  \frame {
		\frametitle{Delopgave 3}
		\framesubtitle{\textbf{Opgave 1} - Find stationær fordeling for Markov-kæden beskrevet ud for den stokastiske matrix $P_1$}
    \fontsize{6pt}{7.2}\selectfont

    \[
    (P_1-I)x=0 \Longleftrightarrow \begin{bmatrix}
      -1 & 0 & 0 & \frac{1}{4} & 0 & \frac{1}{2} & \frac{1}{3} & 0\\
      0 & -1 & 0 & \frac{1}{4} & 0 & 0 & 0 & 0 \\
      0 & \frac{1}{2} & -1 & 0 & 1 & 0 & 0 & 0 \\
      1 & \frac{1}{2} & 0 & -1 & 0 & \frac{1}{2} & \frac{1}{3} & 0\\
      0 & 0 & 1 & 0 & -1 & 0 & \frac{1}{3} & 0\\\mathsf{}
      0 & 0 & 0 & \frac{1}{4} & 0 & -1 & 0 & 0\\
      0 & 0 & 0 & \frac{1}{4} & 0 & 0 & -1 & 0
    \end{bmatrix}
    =
    \begin{bmatrix}
      1 & 0 & 0 & 0 & 0 & 0 & 0 & 0 \\
      0 & 1 & 0 & 0 & 0 & 0 & 0 & 0 \\
      0 & 0 & 1 & 0 & -1 & 0 & 0 & 0 \\
      0 & 0 & 0 & 1 & 0 & 0 & 0 & 0 \\
      0 & 0 & 0 & 0 & 0 & 1 & 0 & 0 \\
      0 & 0 & 0 & 0 & 0 & 0 & 1 & 0 \\
      0 & 0 & 0 & 0 & 0 & 0 & 0 & 0 \\
    \end{bmatrix}
    \]

    \[w=\begin{bmatrix} 0\\0\\1\\0\\1\\0\\0\end{bmatrix},q=\begin{bmatrix} 0\\0\\0.5\\0\\0.5\\0\\0\end{bmatrix}\]

	}

  \frame {
		\frametitle{Delopgave 3}
		\framesubtitle{\textbf{Opgave 3} - Er der entydig stationær fordeling i de to tilfælde?}

    \begin{itemize}
      \item Det kræves at den stokastiske matrix er \textbf{regular}
      \item Den er kun regular hvis alle elementer i en a $P^{(k)}$ potens kun indeholder non-zero entries.
      \item Ingen af de to tilfælde har en entydig stationær fordeling.
    \end{itemize}

	}

  \frame {
		\frametitle{Delopgave 4}
		\framesubtitle{\textbf{Opgave 1} - Hvorfor vil en Markov-kæde med stokastisk matrix $P$ altid have en entydig stationær fordeling hvis $0<\alpha<1$?}

    \[
	P=\alpha \begin{bmatrix}
			0 & 0 & 0 & \frac{1}{4} & 0 & \frac{1}{2} & \frac{1}{3} \\
			0 & 0 &0 & \frac{1}{4} & 0 & 0 & 0 \\
			0 & \frac{1}{2} & 0 &0 & 1 & 0 & 0 \\
			1 & \frac{1}{2} & 0 &0 & 0 & \frac{1}{2} & \frac{1}{3} \\
			0 & 0 & 1 & 0 & 0 & 0 & \frac{1}{3} \\
			0 & 0 & 0 & \frac{1}{4} & 0 & 0 &0 \\
			0 & 0 & 0 & \frac{1}{4} & 0 & 0 &0
		\end{bmatrix}+(1-\alpha)\begin{bmatrix}
		\frac{1}{7} & \frac{1}{7} & \frac{1}{7} & \frac{1}{7} & \frac{1}{7} & \frac{1}{7} & \frac{1}{7} \\
		\frac{1}{7} & \frac{1}{7} & \frac{1}{7} & \frac{1}{7} & \frac{1}{7} & \frac{1}{7} & \frac{1}{7} \\
		\frac{1}{7} & \frac{1}{7} & \frac{1}{7} & \frac{1}{7} & \frac{1}{7} & \frac{1}{7} & \frac{1}{7} \\
		\frac{1}{7} & \frac{1}{7} & \frac{1}{7} & \frac{1}{7} & \frac{1}{7} & \frac{1}{7} & \frac{1}{7} \\
		\frac{1}{7} & \frac{1}{7} & \frac{1}{7} & \frac{1}{7} & \frac{1}{7} & \frac{1}{7} & \frac{1}{7} \\
		\frac{1}{7} & \frac{1}{7} & \frac{1}{7} & \frac{1}{7} & \frac{1}{7} & \frac{1}{7} & \frac{1}{7} \\
		\frac{1}{7} & \frac{1}{7} & \frac{1}{7} & \frac{1}{7} & \frac{1}{7} & \frac{1}{7} & \frac{1}{7} \\
	\end{bmatrix}
	\]

    \begin{itemize}
      \item Når $0<\alpha<1$ så vil alle indgange/entries være strengt/strictly positive og dermed er $P$ en regulær stokastisk matrix. $P^1$ indeholder nemlig kun positive tal hvilket ved definition i Theorem 18 derfor har en entydig stationær fordeling.
    \end{itemize}

	}

  \frame {
		\frametitle{Delopgave 4}
		\framesubtitle{\textbf{Opgave 2} - Forklar hvad et lille/sort $\alpha$ betyder for vores vurdering af surferens adfærd?}

    \[
	P=\alpha \begin{bmatrix}
			0 & 0 & 0 & \frac{1}{4} & 0 & \frac{1}{2} & \frac{1}{3} \\
			0 & 0 &0 & \frac{1}{4} & 0 & 0 & 0 \\
			0 & \frac{1}{2} & 0 &0 & 1 & 0 & 0 \\
			1 & \frac{1}{2} & 0 &0 & 0 & \frac{1}{2} & \frac{1}{3} \\
			0 & 0 & 1 & 0 & 0 & 0 & \frac{1}{3} \\
			0 & 0 & 0 & \frac{1}{4} & 0 & 0 &0 \\
			0 & 0 & 0 & \frac{1}{4} & 0 & 0 &0
		\end{bmatrix}+(1-\alpha)\begin{bmatrix}
		\frac{1}{7} & \frac{1}{7} & \frac{1}{7} & \frac{1}{7} & \frac{1}{7} & \frac{1}{7} & \frac{1}{7} \\
		\frac{1}{7} & \frac{1}{7} & \frac{1}{7} & \frac{1}{7} & \frac{1}{7} & \frac{1}{7} & \frac{1}{7} \\
		\frac{1}{7} & \frac{1}{7} & \frac{1}{7} & \frac{1}{7} & \frac{1}{7} & \frac{1}{7} & \frac{1}{7} \\
		\frac{1}{7} & \frac{1}{7} & \frac{1}{7} & \frac{1}{7} & \frac{1}{7} & \frac{1}{7} & \frac{1}{7} \\
		\frac{1}{7} & \frac{1}{7} & \frac{1}{7} & \frac{1}{7} & \frac{1}{7} & \frac{1}{7} & \frac{1}{7} \\
		\frac{1}{7} & \frac{1}{7} & \frac{1}{7} & \frac{1}{7} & \frac{1}{7} & \frac{1}{7} & \frac{1}{7} \\
		\frac{1}{7} & \frac{1}{7} & \frac{1}{7} & \frac{1}{7} & \frac{1}{7} & \frac{1}{7} & \frac{1}{7} \\
	\end{bmatrix}
	\]

    \begin{itemize}
      \item Et lille $\alpha$ vil betyde at der er stor sandsynlighed for at brugeren blot indtaster en ny internetadresse
      i stedet for at følge links på siden.
      \item Et stort $\alpha$ må så betyde at der er stor sandsynlighed for at brugeren følger links på siden.
    \end{itemize}

	}

  \frame {
		\frametitle{Delopgave 4}
		\framesubtitle{\textbf{Opgave 3} - Beregning af den entydige stationære fordeling for de to eksempler i Delopgave 3. Hvordan vil jeg "ranke" internetsiderne?}
    \fontsize{10pt}{7.2}\selectfont

    \[ P=\alpha P_1+(1-\alpha)P_2 \]

    \begin{align*}
      \begin{bmatrix}
        1 & 0 & 0 & 0 & 0 & 0 & -1.7083 & 0 \\
        0 & 1 & 0 & 0 & 0 & 0 & -1 & 0 \\
        0 & 0 & 1 & 0 & 0 & 0 & -4.8769 & 0 \\
        0 & 0 & 0 & 1 & 0 & 0 & -2.9570 & 0 \\
        0 & 0 & 0 & 0 & 1 & 0 & -4.8003 & 0 \\
        0 & 0 & 0 & 0 & 0 & 1 & -1 & 0 \\
        0 & 0 & 0 & 0 & 0 & 0 & 0 & 0
      \end{bmatrix}
    \end{align*}

    \[\text{Inlusiv de røde pile: }\mathbf{q}=
	 \begin{bmatrix}
		 1.7083/17.3425 \\ 1.000/17.3425 \\ 4.8796/17.3425 \\ 2.9570/17.3425 \\ 4.8003/17.3425 \\ 1/17.3425 \\ 1/17.3425
	 \end{bmatrix}
	 =
	 \begin{bmatrix}
		0.0985048\\0.05766135\\0.28120974\\0.17050718\\0.27679424\\0.05766135\\0.05766135
	\end{bmatrix}\]
	}
  \frame {
		\frametitle{Delopgave 4}
		\framesubtitle{\textbf{Opgave 3} - Beregning af den entydige stationære fordeling for de to eksempler i Delopgave 3. Hvordan vil jeg "ranke" internetsiderne? (fortsættelse)}

    \begin{itemize}
      \item Uden de røde pile og ligeligt fordelt sandsynlighed:
    \end{itemize}

    \begin{align}
      \mathbf{q}=\begin{bmatrix}
        7.1058/46.6667 \\ 4.6883/46.6667 \\ 7.9685/46.6667 \\ 13.0176/46.6667 \\ 8.1982/46.6667 \\ 4.6883/46.6667 \\ 1/46.6667
      \end{bmatrix}=\begin{bmatrix}
        0.15 \\ 0.10 \\ 0.17 \\ 0.28 \\ 0.18 \\ 0.10 \\ 0.02
      \end{bmatrix}
    \end{align}
	}
\end{document}